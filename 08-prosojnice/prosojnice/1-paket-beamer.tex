%  Naslov prosojnice lahko naredimo tudi z dodatnim parametrom okolja `frame`.
\begin{frame}{Posebnosti prosojnic}
	% Naloga 2.3.1:
	% Dodajte ukaze, ki bodo poskrbeli, da se bo prosojnica odkrivala postopoma,
	% tako kot v datoteki prosojnice-resitev.pdf

	Za prosojnice je značilna uporaba okolja \texttt{frame},
	s katerim definiramo posamezno prosojnico,
	\pause
	postopno odkrivanje prosojnic,
	\pause
	ter nekateri drugi ukazi, ki jih najdemo v paketu \texttt{beamer}.
	\pause
	\begin{exampleblock}{Primer}
		Verjetno ste že opazili, da za naslovno prosojnico niste uporabili
		ukaza \texttt{maketitle}, ampak ukaz \texttt{titlepage}.
	\end{exampleblock}
\end{frame}

\begin{frame}{Poudarjeni bloki}
	% Naloga 2.3.2:
	% Oblikujte poudarjena bloka z opombo in opozorilom.
	\begin{block}{Opomba}
		Okolja za poudarjene bloke so \texttt{block}, \texttt{exampleblock} in \texttt{alertblock}.
	\end{block}
		
	\begin{alertblock}{Pozor}
		Začetek poudarjenega bloka (ukaz \texttt{begin}) vedno sprejme 
		dva parametra: okolje in naslov bloka.
		Drugi parameter (za naslov) je lahko prazen.		
	\end{alertblock}

\end{frame}

\begin{frame}{Tudi v predstavitvah lahko pišemo izreke in dokaze}
	% Naloga 2.3.2:
	% Oblikujte okolje itemize, tako da se bo njegova vsebina postopoma odkrivala.
	% Ne smete uporabiti ukaza `pause'.
	% Beseda `največje' naj bo poudarjena šele na četrti podprosojnici.

	\begin{izrek}
	   Praštevil je neskončno mnogo.
	\end{izrek}
	\begin{proof}
	   Denimo, da je praštevil končno mnogo.
	   	% S pomožnim parametrom <+-> lahko določimo, da se bodo 
		% elementi naštevanja odkrivali postopoma.
	   \begin{itemize}[<+->]
		  \item Naj bo $p$ \alert{največje} praštevilo.
		  \item Naj bo $q$ produkt števil $1$, $2$, \ldots, $p$.
		  \item Število $q+1$ ni deljivo z nobenim praštevilom, torej je $q+1$ praštevilo.
		  \item To je protislovje, saj je $q+1>p$. \qedhere
	   \end{itemize}
	\end{proof}
 \end{frame}
 
