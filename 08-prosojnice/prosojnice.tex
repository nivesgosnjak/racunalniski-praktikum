\documentclass{article}
% Naloga 1.3.1: Za dokument uporabite razred `beamer'.
% Ne dodajajte nastavitve za velikost pisave, kot je bila v datoteki `5-prosojnice.tex`.

% Naloga 1.3.2: vključite paket `predavanja'.

% Naloga 1.3.3: definirajte okolji `definicija' in `izrek'.
% Namig: z iskanjem po datotekah (Ctrl+Shift+F oz. Cmd+Shift+F) 
% poiščite niz `{definicija}' ali niz `{izrek}'.

\begin{document}

% Naloga 1.3.4: pripravite naslovno stran z vsebino:
% - naslov: Matematični izrazi in uporaba paketa \texttt{beamer}
% - podnaslov: \emph{Matematičnih} nalog ni treba reševati!
% - inštitut: Fakulteta za matematiko in fiziko
% - datum: naj se ne izpiše; to dosežete z ukazom \date{}.
% Zgornje podatke nastavite z ukazi kot v dokumentih razreda `article`.
% Več o tem, kako se naredi naslovno stran, si preberite na naslovu na naslovu:
% https://www.overleaf.com/learn/latex/Beamer
% To stran preberite do vključno razdelka "Creating a table of contents".
% Ukaz `\titlepage` deluje podobno kot ukaz `\maketitle`, ki ste ga že srečali.

% Naloga 1.3.5: pripravite kazalo vsebine.
% 1. Naslov prosojnice, s kazalom vsebine naj bo "Kratek pregled"
% 2. S pomožnim parametrom `pausesections' (v oglatih oklepajih) 
%    določite, da naj se kazalo vsebine odkriva postopoma.
%    Poglejte, kako deluje ta ukaz.
% 3. Ker ni videti preveč lepo, pomožni parameter zakomentirajte.

\section{Paket \texttt{beamer}}

\section{Paketa \texttt{amsmath} in \texttt{amsfonts}}

\section[Matematika, 1. del\\\large{Analiza, logika, množice}]{Matematika, 1. del}
\begin{frame}{Logika in množice}
	\begin{enumerate}
		\item
		Poišči preneksno obliko formule 
		$$\exists x : P(x) \wedge \forall x : Q(x) \Rightarrow \forall x : R(x)$$
		\item 
		Definiramo množici $A=[2,5]$ in $B=\{0,1,2,3,4\ldots\}$.
		V ravnino nariši:
		\begin{enumerate}
		   \item $A\cap B \times \emptyset$
		   \item $(A\cap B)^c \times$
		\end{enumerate}
		\item
		Dokaži:
		\begin{itemize}
			\item $(A\Rightarrow B)\sim(\neg B\Rightarrow \neg A))$
			\item $\neg(A \vee B)\sim \neg A \wedge \neg BS$
		\end{itemize}
	\end{enumerate}
\end{frame}

\begin{frame}{Analiza}
	\begin{enumerate}
		\item
		Pokaži, da je funkcija $x\mapsto \sqrt{x}$ enakomerno zvezna na $[0,\infty)$.
		\item 
		Katero krivuljo določa sledeč parametričen zapis?
		% Spodaj si pomagajte z dokumentacijo o razmikih v matematičnem načinu.
		% https://www.overleaf.com/learn/latex/Spacing_in_math_mode
		$$
		   x(t) = a \cos t, ?? % tu manjka ukaz za presledek
		   y(t) = b \sin t, ?? % tu manjka ukaz za presledek
		   t \in [0, 2 \pi]
		$$ 
		\item
		Pokaži, da ima $f(x)=3x + sin(2x)$ inverzno funkcijo 
		in izračunaj $(f^{-1})'(3\pi)$.
		
		\item
		Izračunaj integral 
		% V rešitvah smo spodnji integral zapisali v vrstičnem načinu,
		% ampak v prikaznem slogu. To naredite tako, da v matematičnem načinu najprej
		% uporabite ukaz displaystyle.
		% Pred dx je presledek: pravi ukaz je \,
		$\displaystyle \int\frac{2+\sqrt{x+1}}{(x+1)^2-\sqrt{x+1}} \ dx$
		%  
		\item 
		Naj bo $g$ zvezna funkcija. Ali posplošeni integral 
		$\int_{0}^{1}\frac{g(x)}{x^2} \ dx$
		konvergira ali divergira? Utemelji.
	\end{enumerate}
\end{frame}

\begin{frame}{Kompleksna števila}
	\begin{enumerate}
		\item
		Naj bo $z$ kompleksno število, $z \ne 1$ in ??.
		Dokaži, da je število \( i \, \frac{z+1}{z-1} \) realno.
		\item
		Poenostavi izraz:
		\[\frac{\frac{3+i}{2-2i}+\frac{7i}{1-i}}{1+\frac{i-1}{4}-\frac{5}{2-3i}}\]
	\end{enumerate}
\end{frame}

\section{Stolpci in slike}

\section{Paket \texttt{beamer} in tabele}

\section[Matematika, 2. del\\\large{Zaporedja, algebra, grupe}]{Matematika, 2. del}

\end{document}