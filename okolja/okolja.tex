\documentclass{article}
\usepackage[utf8]{inputenc}
\usepackage[T1]{fontenc}
\usepackage{lmodern}
\usepackage{amsmath}
\usepackage{amsthm}
\usepackage{url}
\usepackage{graphicx}
\usepackage{booktabs}
\usepackage{multirow}
%We can define the environment in the preamble
\newenvironment{mijau}[2][This is a box]
    {\begin{center}
    Argument 1 (\#1)=#1\\[1ex]
    \begin{tabular}{|p{0.9\textwidth}|}
    \hline\\
    Argument 2 (\#2)=#2\\[2ex]
    }
    { 
    \\\\\hline
    \end{tabular} 
    \end{center}
    }
\newenvironment{magic}[3]
    {\begin{table}
    \centering
    \caption{#1}
    \label{#2}
    \begin{tabular}{|*{#3}{c|}}
        \hline
    }
    {\end{tabular}
    \end{table}
    }

\begin{document}
\textbf{Example 1}: Use the default value for the first argument:
 
\begin{mijau}{Some preliminary text}
This text is \textit{inside} the environment.
\end{mijau}

This text is \textit{outside} the environment.

\vskip12pt

\textbf{Example 2}: Provide a value for the first argument:
 
\begin{mijau}[This is not the default value]{Some more preliminary text}
This text is still \textit{inside} the environment.
\end{mijau}

This text is also \textit{outside} the environment.

\begin{magic}{mijau}{tab:3}{3}
    3 & 6 & 6 \\\hline
    3 & 6 & 6 \\\hline
    3 & 6 & 6 \\\hline
\end{magic}
\begin{boxed}
{blah}
\end{boxed}

\end{document}